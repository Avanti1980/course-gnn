\documentclass{ctexart}
\usepackage{avanti}
\everymath{\color{Solarized-magenta}}
\pagestyle{empty} % 没有页眉和页脚

\tikzset{font=\Large}

\tikzset{base/.style = {smooth, thick, draw=Solarized-base01}}
\tikzset{nc/.style = {circle, minimum height=0.8cm, base}}
\tikzset{label/.style = {Solarized-base01, right}}

\begin{document}

\begin{tikzpicture}

    \pgfmathsetmacro{\r}{2}; % 五边形中心到顶点的距离
    \pgfmathsetmacro{\inc}{1};

    \path (-2, 2.5) coordinate (p);
    \path (p) ++(3.5,0) coordinate (d1);
    \path (d1) ++(2.5,0) coordinate (d2);
    \path (d2) ++(2.8,0) coordinate (equ);
    \path (equ) ++(1.8,0) coordinate (s);
    \path (s) ++(4.3,0) coordinate (g);

    \path (d1) ++(0,\inc) coordinate (d3);
    \path (d1) ++(0,-\inc) coordinate (d5);
    \path (d2) ++(0,\inc) coordinate (d4);
    \path (d2) ++(0,-\inc) coordinate (d6);

    \path (g) ++(18:\r) coordinate (g1);
    \path (g) ++(90:\r) coordinate (g2);
    \path (g) ++(162:\r) coordinate (g3);
    \path (g) ++(234:\r) coordinate (g4);
    \path (g) ++(306:\r) coordinate (g5);

    \node[label] at (p) {加权成对数据};

    \node[label] at (d3) {$(a,b): 1$};
    \node[label] at (d4) {$(a,d): 2$};
    \node[label] at (d1) {$(a,e): 3$};
    \node[label] at (d2) {$(b,c): 4$};
    \node[label] at (d5) {$(c,e): 5$};
    \node[label] at (d6) {$(d,e): 6$};

    \node[label] at (equ) {$\Longleftrightarrow$};

    \node[label] at (s) {加权图};

    \node[nc] (1) at (g1) {$a$};
    \node[nc] (2) at (g2) {$b$};
    \node[nc] (3) at (g3) {$c$};
    \node[nc] (4) at (g4) {$d$};
    \node[nc] (5) at (g5) {$e$};

    \draw (1) -- (2) node [midway,above=10pt,right] {$1$};
    \draw (1) -- (4) node [midway,above=4pt] {$2$};
    \draw (1) -- (5) node [midway,below,right] {$3$};
    \draw (2) -- (3) node [midway,above=10pt,left] {$4$};
    \draw (3) -- (5) node [midway,above=16pt,left] {$5$};
    \draw (4) -- (5) node [midway,below] {$6$};

    \draw[base] (1) -- (2) -- (3) -- (5) -- (1) -- (4) -- (5);

\end{tikzpicture}


\end{document}