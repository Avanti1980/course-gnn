\documentclass{ctexart}
\usepackage{avanti}
\everymath{\color{Solarized-magenta}}
\pagestyle{empty} % 没有页眉和页脚

\tikzset{font=\Large}

\tikzset{base/.style = {smooth, thick, draw=Solarized-base01}}
\tikzset{arrow/.style={->,>=stealth,base}}
\tikzset{nc/.style = {circle, minimum height=0.8cm, base}}
\tikzset{label/.style = {Solarized-base01, right}}

\begin{document}

\begin{tikzpicture}

    \pgfmathsetmacro{\r}{1.5}; % 五边形中心到顶点的距离
    \pgfmathsetmacro{\inc}{1};

    \path (-2, 2.5) coordinate (p);
    \path (p) ++(3.5,0) coordinate (d1);
    \path (d1) ++(1.5,0) coordinate (d2);
    \path (d2) ++(2,0) coordinate (equ);
    \path (equ) ++(1.8,0) coordinate (s);
    \path (s) ++(3.8,0) coordinate (g);

    \path (d1) ++(0,\inc) coordinate (d3);
    \path (d1) ++(0,-\inc) coordinate (d5);
    \path (d2) ++(0,\inc) coordinate (d4);
    \path (d2) ++(0,-\inc) coordinate (d6);

    \path (g) ++(18:\r) coordinate (g1);
    \path (g) ++(90:\r) coordinate (g2);
    \path (g) ++(162:\r) coordinate (g3);
    \path (g) ++(234:\r) coordinate (g4);
    \path (g) ++(306:\r) coordinate (g5);

    \node[label] at (p) {有向成对数据};

    \node[label] at (d3) {$\langle a,b \rangle$};
    \node[label] at (d4) {$\langle a,d \rangle$};
    \node[label] at (d1) {$\langle a,e \rangle$};
    \node[label] at (d2) {$\langle b,c \rangle$};
    \node[label] at (d5) {$\langle c,e \rangle$};
    \node[label] at (d6) {$\langle d,e \rangle$};

    \node[label] at (equ) {$\Longleftrightarrow$};

    \node[label] at (s) {有向图};

    \node[nc] (1) at (g1) {$a$};
    \node[nc] (2) at (g2) {$b$};
    \node[nc] (3) at (g3) {$c$};
    \node[nc] (4) at (g4) {$d$};
    \node[nc] (5) at (g5) {$e$};

    \draw[arrow] (1) -- (2);
    \draw[arrow] (1) -- (4);
    \draw[arrow] (1) -- (5);
    \draw[arrow] (2) -- (3);
    \draw[arrow] (3) -- (5); 
    \draw[arrow] (4) -- (5);

\end{tikzpicture}


\end{document}