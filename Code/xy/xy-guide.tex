\documentclass[openany]{ctexart}
\usepackage{amsmath,amsthm,amssymb,amscd}
\usepackage{fullpage}
\usepackage{graphicx}
\usepackage{epsfig}
\usepackage{subfig} 
\usepackage{algorithm,algorithmic}
\usepackage[bookmarksnumbered,bookmarksopen,colorlinks=true,citecolor=green,anchorcolor=blue,linkcolor=blue,CJKbookmarks=true]{hyperref}
\usepackage{tikz}
\usepackage{bm}
\usepackage{listings} 
%\usepackage[cache=false]{minted}
\usepackage{xcolor,framed}
\pagecolor[rgb]{0.9, 0.99, 0.9}
\usepackage{adjustbox}
\usepackage[all]{xy}
\usepackage{enumitem}
\setitemize{itemsep=0pt,partopsep=0pt,parsep=0pt,topsep=0pt}
\setenumerate{itemsep=0pt,partopsep=0pt,parsep=0pt,topsep=0pt}
\setdescription{itemsep=0pt,partopsep=0pt,parsep=0pt,topsep=0pt}

%\setlength{\parskip}{0.3em}

\usepackage{fontspec} 
\setCJKmainfont[BoldFont=FZHei-B01,ItalicFont=FZHei-B01]{方正苏新诗柳楷简体-yolan}

\theoremstyle{definition}
\newtheorem{thm}{定理}
\newtheorem{prop}{命题}
\newtheorem{lem}{引理}
\newtheorem{cor}{推论}
\newtheorem{defn}{定义}
\newtheorem{exam}{例}
\newtheorem*{exc}{习题}
\newtheorem*{ans}{解答}
\newtheorem*{rmk}{注}
%\newtheorem{Proof}{\bf{证明}}
\renewcommand{\proofname}{证明}
\renewcommand{\qedsymbol}{M.X.}

%\theoremsymbol{M.X.}

\def \i {\mathbf{i}}
\def \j {\mathbf{j}}
\def \k {\mathbf{k}}

\def \zerov {\bm{0}}
\def \av {\bm{a}}
\def \bv {\bm{b}}
\def \cv {\bm{c}}
\def \dv {\bm{d}}
\def \ev {\bm{e}}
\def \fv {\bm{f}}
\def \gv {\bm{g}}
\def \hv {\bm{h}}
\def \pv {\bm{p}}
\def \qv {\bm{q}}
\def \uv {\bm{u}}
\def \vv {\bm{v}}
\def \wv {\bm{w}}
\def \xv {\bm{x}}
\def \yv {\bm{y}}
\def \zv {\bm{z}}

\def \Av {\bm{A}}
\def \Bv {\bm{B}}
\def \Cv {\bm{C}}
\def \Dv {\bm{D}}
\def \Ev {\bm{E}}
\def \Fv {\bm{F}}
\def \Gv {\bm{G}}
\def \Hv {\bm{H}}
\def \Iv {\bm{I}}
\def \Jv {\bm{J}}
\def \Kv {\bm{K}}
\def \Mv {\bm{M}}
\def \Pv {\bm{P}}
\def \Qv {\bm{Q}}
\def \Sv {\bm{S}}
\def \Uv {\bm{U}}
\def \Vv {\bm{V}}
\def \Wv {\bm{W}}

\def \alphav {\bm{\alpha}}
\def \lambdav {\bm{\lambda}}
\def \epsilonv {\bm{\epsilon}}
\def \xiv {\bm{\xi}}
\def \muv {\bm{\mu}}
\def \nuv {\bm{\nu}}
\def \phiv {\bm{\phi}}
\def \Phiv {\bm{\Phi}}
\def \Lambdav {\bm{\Lambda}}

\def \Bcal {\mathcal{B}}
\def \Ccal {\mathcal{C}}
\def \Fcal {\mathcal{F}}

\def \Abb {\mathbb{A}}
\def \Dbb {\mathbb{D}}
\def \Ebb {\mathbb{E}}
\def \Pbb {\mathbb{P}}
\def \Qbb {\mathbb{Q}}
\def \Rbb {\mathbb{R}}
\def \Sbb {\mathbb{S}}
\def \Zbb {\mathbb{Z}}

\def \st {\mathrm{s.t.}}
\def \sign {\mathrm{sign}}
\def \diff {\mathrm{d}}
\def \mod {\mathrm{mod}}
\def \aut {\mathrm{Aut}}
\def \per {\mathrm{Per}}

\DeclareMathOperator*{\argmin}{argmin}
\DeclareMathOperator*{\argmax}{argmax}

\allowdisplaybreaks[4]

\begin{document}
\title{\textbf{XY-pic使用手册}}
\author{圆眼睛的阿凡提哥哥}
\date{\today}
\maketitle

\begin{exam}
\begin{lstlisting}
\xymatrix{
U \ar@/_/[ddr]_y \ar@/^/[drr]^x \ar@{.>}[dr]|-{(x,y)} \\
& X \times_Z Y \ar[d]^q \ar[r]_p & X \ar[d]_f \\
& Y \ar[r]^g & Z \\
}
\end{lstlisting}

\begin{align*}
\xymatrix{
U \ar@/_/[ddr]_y \ar@/^/[drr]^x \ar@{.>}[dr]|-{(x,y)} \\
& X \times_Z Y \ar[d]^q \ar[r]_p & X \ar[d]_f \\
& Y \ar[r]^g & Z \\
}
\end{align*}
\end{exam}


\begin{exam}
\begin{lstlisting}
\xymatrix{
A &*+[F]{\sum_{i=n}^m {i^2}} \\
& {\bullet} & D \ar[ul]
}
\end{lstlisting}

\begin{align*}
\xymatrix{
A & *+[F]{\sum_{i=n}^m {i^2}} \\
& {\bullet} & D \ar[ul]
}
\end{align*}
\end{exam}

箭头的模板是``$\backslash ar @style [direction] label$'',其中$@style$是对箭头样式的描述,可以省略,默认就如(\ref{eq: direction})式中所示;$direction$是$l$、$r$、$u$、$d$组成的字符串,控制箭头指向的方向:
\begin{align} \label{eq: direction}
\begin{split}
\xymatrix{
\backslash ar[ull] & \backslash ar[ul] & \backslash ar[u] & \backslash ar[ur] & \backslash ar[urr] \\
\backslash ar[ll] & & A \ar[ull] \ar[ul] \ar[u] \ar[ur] \ar[urr] \ar[ll] \ar[rr] \ar[dll] \ar[dl] \ar[d] \ar[dr] \ar[drr] & & \backslash ar[rr] \\
\backslash ar[dll] & \backslash ar[dl] & \backslash ar[d] & \backslash ar[dr] & \backslash ar[drr]
} 
\end{split}
\end{align}
$@style$的模板是$@variant\{tail~shaft~head\}$,其中
\begin{itemize}
\item $head$和$tail$分别是箭头和箭尾,可取的值完全一样,默认就是(\ref{eq: direction})式中的箭头:
\begin{align*}
\xymatrix{
@\{>\} & @\{>>\} & @\{>|\} & @\{>>|\} & @\{)\} \\
@\{/\} & & A \ar@{>}[ull] \ar@{>>}[ul] \ar@{>|}[u] \ar@{>>|}[ur] \ar@{)}[urr] \ar@{/}[ll] \ar@{//}[rr] \ar@{x}[dll] \ar@{+}[dl] \ar@{|}[d] \ar@{||}[dr] \ar@{o}[drr] & & @\{//\} \\
@\{x\} & @\{+\} & @\{|\} & @\{||\} & @\{o\} 
}
\end{align*}

\item $shaft$是箭身,默认就是(\ref{eq: direction})式中的单直线,此外还可以取$=$、$:$、$\~{}$、$\~{}\~{}$、$.$、$--$:
\begin{align*}
\xymatrix{
@\{=>\} & & @\{:>\} \\
@\{\~{}>\} & A \ar@{=>}[ul] \ar@{:>}[ur] \ar@{~>}[l] \ar@{~~>}[r] \ar@{.>}[dl] \ar@{-->}[dr] & @\{\~{}\~{}>\} \\
@\{.>\} & & @{}\{-->\}
} 
\end{align*}

\item $variant$可以省略,默认如(\ref{eq: direction})式中所示,此外可取值$\^{}$、$\_{}$、$2$、$3$:
\begin{align*}
\xymatrix{
@\^{}\{<->\} & A \ar @^{<->} [l] \ar @_{<->} [dl] \ar @{<->} [d] \ar @2{<->} [dr] \ar @3{<->} [r] & @3\{<->\} \\
@\_{}\{<->\} & @\{<->\} & @2{}\{<->\}
} 
\end{align*}
其中$@2\{-\}$与$@\{=\}$效果相同、$@2\{.\}$与$@\{:\}$效果相同:
\begin{align*}
\xymatrix{
@\{=>\} & A \ar@{=>}[l] \ar @2{->}[r] & @2\{->\} \\
@\{:>\} & A \ar@{:>}[l] \ar @2{.>}[r] & @2\{.>\} \\
} 
\end{align*}
$variant$默认是作用在整个$\{tail~shaft~head\}$上,也可以将其写到花括号里面:$variant\{tail\}$、$variant\{shaft\}$、$variant\{head\}$,只作用于局部:
\begin{align*}
\xymatrix{
A \ar@{^{(}3._{|}}[r] & @\{\^{}\{(\}3.\_\{|\}\}
}
\end{align*}

\item $label$是箭头上的标记,可以省略,$\^{}$表示是箭头上方标记,$\_{}$表示是箭头下方标记,$|$表示是含在箭身里的标记:
\begin{align*}
\xymatrix{
A \ar@{>}[r]^u_d|m & @\{>\}[r]\^{}u\_{}d|m
}
\end{align*}
其中标记的默认位置是在箭头两边元素的中点连线的中点,如果两边元素的宽度差别较大,就会出现上式这种情况,此时需要调整标记的位置:
\begin{itemize}
\item 通过$</>$将标记设置在箭头的起始/结束位置:
\begin{align*}
\xymatrix{
A \ar@{>}[r]^<{u}_>{d} & @\{>\}[r]\^{}<\{u\}\_{}>\{d\}
}
\end{align*}

\item 通过$<</>>$将标记设置在离起始/结束位置差一点的位置:
\begin{align*}
\xymatrix{
A \ar@{>}[r]^<<{u}_>>{d} & @\{>\}[r]\^{}<<\{u\}\_{}>>\{d\}
}
\end{align*}

\item 通过百分比来设置标记的位置,起始和结束位置默认是箭头两边元素的中点:
\begin{align*}
\xymatrix{
A \ar@{>}[r]^(0){u}_(1){d}|(.15)m & @\{>\}[r]\^{}(0)\{u\}\_{}(1)\{d\}|(.15)m
}
\end{align*}
也可以在百分比前加上$<$或$>$将起始/结束位置改为箭头的起始/结束位置:
\begin{align*}
\xymatrix{
A \ar@{>}[r]^<(.15){u}_>(.8){d}|<>(.4)m & @\{>\}[r]\^{}<(.15)\{u\}\_{}>(.8)\{d\}|<>(.4)m
}
\end{align*}

\item 通过$|\backslash hole$可以让箭头在中间断掉,此外通过$|!\{[s];[t]\}\backslash hole$可以设置在与其他箭头交叉的地方断开,其中$s$和$t$分别是其他箭头的起始/结束元素相对于当前元素的方向:
\begin{align*}
\xymatrix{
A \ar[dr]|\hole \\
& \backslash ar[dr] | \backslash hole
} \qquad
\xymatrix{
A \ar[drr]|!{[d];[r]}\hole & \backslash ar [ur] \\
A \ar[ur] & & \backslash ar[drr]|!\{[d];[r]\} \backslash hole
}
\end{align*}
也可以通过$\backslash ar@\{\}[dr] |=$设置箭头为空达到只画标记的目的:
\begin{align*}
\xymatrix{
A \ar@{->}[r] \ar@{->}[d] \ar@{}[dr]|= & B \ar@{->}[d] \\ B \ar@{->}[r] & C
}
\end{align*}

\end{itemize}

\end{itemize}

标记和元素也可以修饰,模板是$*modifiers\{text\}$,其中$modifiers$用来改变目标的形状和大小,最常用的几个如下:
\begin{align*}
\xymatrix@1{
\text{*+<5pt>[o][F]\{a\}} & \text{*+<5pt>[F]\{a\}} & \text{*+<5pt>[F.]\{a\}} \\
*+<5pt>[o][F]{a} & *+<5pt>[F]{a} & *+<5pt>[F.]{a} \\ 
\text{*+<5pt>[F-{}-]\{abcde\}} & \text{*+<5pt>[F-,]\{abcde\}} & \text{*+<5pt>[F-:<3pt>]\{abcde\}} \\
*+<5pt>[F--]{abcde} & *+<5pt>[F-,]{abcde} & *+<5pt>[F-:<3pt>]{abcde} \\
}
\end{align*}

dddddddd
\begin{align*}
\xymatrix@1{
A \ar @{<*\composite{{+}*{()}}>} [rrr] ^*+\txt{High\\label} & & & B
}
\end{align*}

\begin{align*}
\xymatrix@1{
A \ar @{<*\frm{=}>} [rrr] ^*+\txt{High\\label} & & & B
}
\end{align*}

每个箭头上还可以加标记:

\begin{align*}
\xy *=<3cm,1cm>\txt{Box}*\frm{-}
!U!R(.5) *\frm{..}*{\bullet} \endxy
\end{align*}

\begin{align*}
\xy
(0,0) *++={A} *\frm{o} ;
(10,7) *++={B} *\frm{o} **\frm{.}
\endxy
\end{align*}

\begin{align*}
\xymatrix@!0{
& \lambda\omega \ar@{-}[rr]\ar@{-}'[d][dd]
& & \lambda C \ar@{-}[dd]
\\
\lambda2 \ar@{-}[ur]\ar@{-}[rr]\ar@{-}[dd]
& & \lambda P2 \ar@{-}[ur]\ar@{-}[dd]
\\
& \lambda\underline\omega \ar@{-}'[r][rr]
& & \lambda P\underline\omega
\\
\lambda{\to} \ar@{-}[rr]\ar@{-}[ur]
& & \lambda P \ar@{-}[ur]
}, \qquad \xymatrix{
& \lambda\omega \ar@{-}[rr]\ar@{-}'[d][dd]
& & \lambda C \ar@{-}[dd]
\\
\lambda2 \ar@{-}[ur]\ar@{-}[rr]\ar@{-}[dd]
& & \lambda P2 \ar@{-}[ur]\ar@{-}[dd]
\\
& \lambda\underline\omega \ar@{-}'[r][rr]
& & \lambda P\underline\omega
\\
\lambda{\to} \ar@{-}[rr]\ar@{-}[ur]
& & \lambda P \ar@{-}[ur]
}
\end{align*}

\begin{align*}
\xymatrix{
    {\circ} \ar `r[d]^a
    `[rr]^b
    `/4pt[rr]^c
    `[rrr]^d
    `[drrr]^e
    [drrr]^f
    & {\circ} & {\circ} & {\circ} \\
    {\circ} & {\circ} & {\circ} & {\circ} }
\end{align*}

\begin{exam}[箭头类型]
模板是:$\backslash ar ~ @style ~ [direction]$ 
\begin{align*}
\xymatrix{
    @\{ - \} \ar @{->} [d] &
    @\{ => \} \ar @{=>} [d] & 
    @\{ .> \} \ar @{.>} [d] &
    @\{ :> \} \ar @{:>} [d] &
    @\{ \sim > \}  \ar @{~>} [d] &
    @\{ --> \} \ar @{-->} [d] & 
    @\{ --> \} \ar @{~~>} [d] & \\
    A & A & A & A & A & A & A \\
} 
\end{align*}
\end{exam}

\begin{exam}[箭头类型]
模板是:$\backslash ar ~ @style ~ [direction]$ 
\begin{align*}
\xymatrix{
    @\{ - \} \ar @3{->>} [d] &
    @\{ => \} \ar @{=>|} [d] & 
    @\{ .> \} \ar @{.>} [d] &
    @\{ :> \} \ar @{:>} [d] &
    @\{ \sim > \}  \ar @2{~>} [d] &
    @\{ --> \} \ar @{.||} [d] & \\
    A & A & A & A & A & A \\
} 
\end{align*}
\end{exam}

\begin{exam}
\begin{align*}
\xymatrix{
\txt{@\{ / \^ / \}} \ar@/^/[d] & 
\text{@\{ / \_ / \}} \ar@/_/[d] &
\text{@\{ / \_1pc / \}} \ar@/_1pc/[d] &
\text{@\{ / \_2pc / \}} \ar@/_2pc/[d] &
\text{@\{ / \_3pc / \}} \ar@/_3pc/[d] &
\\
A & A & A & A & A \\
} 
\end{align*}
\end{exam}

\begin{exam} 
\begin{align*}
\xymatrix@1{
\text{x} \ar@(ul,dl)|{id} \ar@/^/[rr]|f & & f(x) \ar@/^/[ll]|{f^{-1}}
}
\end{align*}
\end{exam}

    

\end{document}